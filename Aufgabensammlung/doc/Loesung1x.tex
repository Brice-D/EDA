\documentclass[11pt]{scrartcl}

    \usepackage[ngerman]{babel}
    \usepackage{ucs}
    \usepackage[utf8x]{inputenc}
    \usepackage[T1]{fontenc}
    \usepackage{listings}
    \usepackage[usenames,dvipsnames]{xcolor}
    \usepackage{pict2e}
    
    \input kvmacros

    % Define VHDL language for code snippets
    \lstdefinelanguage{VHDL}{
        morekeywords=[1]{
          library,use,all,entity,is,port,in,out,end,architecture,of,
          begin,and,or,Not,downto,ALL
        },
        morekeywords=[2]{
          STD_LOGIC_VECTOR,STD_LOGIC,IEEE,STD_LOGIC_1164,
          NUMERIC_STD,STD_LOGIC_ARITH,STD_LOGIC_UNSIGNED,std_logic_vector,
          std_logic
        },
        morecomment=[l]--
     }
     \colorlet{keyword}{blue!100!black!80}
     \colorlet{STD}{Lavender}
     \colorlet{comment}{green!80!black!90}
     \lstdefinestyle{vhdl}{
        language     = VHDL,
        basicstyle   = \footnotesize \ttfamily,
        keywordstyle = [1]\color{keyword}\bfseries,
        keywordstyle = [2]\color{STD}\bfseries,
        commentstyle = \color{comment},
        breaklines   = true,
        tabsize      = 4,
        frame        = single
     }
     % End of VHDL definition

     % Define underline with color in formulas
    \newsavebox\MBox
    \newcommand\Cline[2][red]{{\sbox\MBox{$#2$}%
        \rlap{\usebox\MBox}\color{#1}\rule[-1.2\dp\MBox]{\wd\MBox}{0.5pt}}}

\begin{document}
\title{Lösungen Aufgabensammlung 3.1.x}
\author{Brice Dorsey Long Dje, Felix Jägers}
\maketitle

\section{Aufgabe 3.1.1}

Gegebene Funktionstabelle f:

\begin{center}
\begin{tabular}{lr}
    \begin{tabular}[t]{r|cccc|c}
  ($i$)\textsubscript{10}&$x$&$y$&$z$&$v$&$f$\\
  \hline
  0&0&0&0&0&0\\
  1&0&0&0&1&0\\
  2&0&0&1&0&1\\
  3&0&0&1&1&1\\
  4&0&1&0&0&1\\
  5&0&1&0&1&1\\
  6&0&1&1&0&0\\
  7&0&1&1&1&1\\
    \end{tabular}
  &
    \begin{tabular}[t]{r|cccc|c}
      ($i$)\textsubscript{10}&$x$&$y$&$z$&$v$&$f$\\
  \hline
  8&1&0&0&0&0\\
  9&1&0&0&1&0\\
  10&1&0&1&0&1\\
  11&1&0&1&1&1\\
  12&1&1&0&0&1\\
  13&1&1&0&1&0\\
  14&1&1&1&0&0\\
  15&1&1&1&1&1\\
    \end{tabular}
\end{tabular}
\end{center}

Aus der Funktionstabelle erstelltes Karnaugh-Diagramme jeweils für DMF und KMF:

\begin{center}
\karnaughmap{4}{$f\textsubscript{DMF}$:}{{$x$}{$y$}{$z$}{$v$}}
{0011110100111001}
{
    \put(1,2){\color{red}\circle{2}}
    \put(2,2){\color{green}\circle{2}}
    \put(3,3.5){\color{blue}\oval(1.9,1)[l]}
    \put(3,3.5){\color{blue}\oval(1.9,1)[r]}
    \put(3.5,0){\color{orange}\oval(1,1.9)[t]}
    \put(3.5,4){\color{orange}\oval(1,1.9)[b]}
}
\karnaughmap{4}{$f\textsubscript{KMF}$:}{{$x$}{$y$}{$z$}{$v$}}
{0011110100111001}
{
    \put(1,4){\color{red}\oval(1.9,1.9)[b]}
    \put(1,0){\color{red}\oval(1.9,1.9)[t]}
    \put(3.5,2){\color{green}\oval(1,1.9)[b]}
    \put(3.5,2){\color{green}\oval(1,1.9)[t]}
    \put(2,0.5){\color{blue}\oval(1.9,1)[l]}
    \put(2,0.5){\color{blue}\oval(1.9,1)[r]}
}
\end{center}

$f\textsubscript{KMF} = (\overline{v} \lor z \lor \overline{x}) \land
 (z \lor y) \land (\overline{z} \lor \overline{y} \lor v) $

\newpage
\section{Aufgabe 3.1.2}
\subsection{a) DNF bestimmen}

Gegebene Funktionstabelle f aus dem Karnaugh-Diagramme abgelesen:

\begin{center}
\begin{tabular}{lr}
    \begin{tabular}[t]{r|cccc|c}
  ($i$)\textsubscript{10}&$x$&$y$&$z$&$v$&$f$\\
  \hline
  0&0&0&0&0&0\\
  1&0&0&0&1&1\\
  2&0&0&1&0&1\\
  3&0&0&1&1&0\\
  4&0&1&0&0&1\\
  5&0&1&0&1&0\\
  6&0&1&1&0&0\\
  7&0&1&1&1&1\\
    \end{tabular}
  &
    \begin{tabular}[t]{r|cccc|c}
      ($i$)\textsubscript{10}&$x$&$y$&$z$&$v$&$f$\\
  \hline
  8&1&0&0&0&1\\
  9&1&0&0&1&0\\
  10&1&0&1&0&0\\
  11&1&0&1&1&1\\
  12&1&1&0&0&0\\
  13&1&1&0&1&1\\
  14&1&1&1&0&1\\
  15&1&1&1&1&0\\
    \end{tabular}
\end{tabular}
\end{center}
\ \\[1em]
Aus der Funktionstabelle die DNF erstellt und direkt die :\\
\ \\
\begin{tabular}{lll}
    $f\textsubscript{DNF} $&$=$&
    $\Cline[red]{(\overline{x} \land \overline{y} \land \overline{z} \land v)} \lor
    \Cline[red]{(\overline{x} \land \overline{y} \land z \land \overline{v})} \lor
    \Cline[green]{(x \land y \land z \land \overline{v})} \lor
    \Cline[green]{(x \land y \land \overline{z} \land v)} \lor$ \\
    &&
    $\Cline[blue]{(\overline{x} \land y \land z \land v)} \lor
    \Cline[blue]{(x \land \overline{y} \land z \land v)} \lor
    \Cline[orange]{(\overline{x} \land y \land \overline{z} \land \overline{v})} \lor
    \Cline[orange]{(x \land \overline{y} \land \overline{z} \land \overline{v})}$\\
\end{tabular}
\subsection{b) Beweis}
Ausklammern der gleichen Variablen innerhabl der jeweils gleich unterstrichenen Terme:\\
\ \\
\begin{tabular}{lll}
    $f\textsubscript{DNF} $&$=$&
    $(\overline{x} \land \overline{y}) \lor
    \Cline[red]{((\overline{z} \land v) \lor (z \land \overline{v}))} \lor
    (x \land y) \lor
    \Cline[green]{((z \land \overline{v}) \lor (\overline{z} \land v))} \lor$\\
    &&
    $(\overline{z} \land \overline{v}) \lor
    \Cline[blue]{((\overline{x} \land y) \lor (x \land \overline{y}))} \lor
    (z \land v) \lor
    \Cline[orange]{((x \land \overline{y}) \lor (\overline{x} \land y))}$
\end{tabular}
\\[1em]
Jeweils der unterstrichene Term ist ein XNOR. Rot und Grün sind gleich und Blau und Orange.
Damit wird das auch direkt zusammen gefasst:\\
\ \\
\begin{tabular}{lll}
    $f\textsubscript{DNF} $&$=$&
    $((\overline{x} \land \overline{y}) \lor (x \land y)) \lor
    z \oplus v \lor
    ((\overline{z} \land \overline{v}) \lor (z \land v)) \lor
    x \oplus y$\\
    &$=$&
    $(\overline{x \oplus y} \lor z \oplus v) \lor
    (x \oplus y \lor  \overline{z \oplus v})$\\
    &$=$&
    $(x \oplus y \land  \overline{z \oplus v}) \lor
    (\overline{x \oplus y} \land z \oplus v)$\\
    &$=$&
    $x \oplus y \oplus z \oplus v$
\end{tabular}
\\[1em]

% VHDL Test code snippet
\begin{lstlisting}[style=vhdl]
-- VHDL test code snippet

library IEEE;
    
\end{lstlisting}
% End VHDL

\end{document}